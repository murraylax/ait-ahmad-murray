\documentclass[english,authoryear,12pt]{elsarticle}
%%%%%%%%%%%%%%%%%%%%%%%%%%%%%%%%%%%%%%%%%%%%%%%%%%%%%%%%%%%%%%%%%%%%%%%%%%%%%%%%%%%%%%%%%%%%%%%%%%%%%%%%%%%%%%%%%%%%%%%%%%%%%%%%%%%%%%%%%%%%%%%%%%%%%%%%%%%%%%%%%%%%%%%%%%%%%%%%%%%%%%%%%%%%%%%%%%%%%%%%%%%%%%%%%%%%%%%%%%%%%%%%%%%%%%%%%%%%%%%%%%%%%%%%%%%%
\usepackage{fullpage}
\usepackage[labelfont=bf,singlelinecheck=false,aboveskip=10pt]{caption}
%\usepackage[font={scriptsize},labelfont={scriptsize},textfont={scriptsize}]{caption}
%\DeclareCaptionFormat{myformat}{#1#2\\#3}
\usepackage{amsmath,graphicx,multicol,chngpage,setspace}
\usepackage{amssymb,amsthm,amstext,amsfonts,enumerate}
\usepackage[colorlinks=true,linkcolor=black, citecolor=blue, urlcolor=blue]{hyperref}
\usepackage[top=1in, bottom=1in, left=1.0in, right=1.0in]{geometry}
\usepackage{lscape}
\usepackage{rotating}
\usepackage{color,float,multirow,array,hyphenat}
\usepackage{booktabs}
\usepackage{subcaption}
\usepackage{natbib} % already loaded by elsarticle
\usepackage{apalike}
\usepackage{babel}
\usepackage{appendix}
\usepackage{afterpage}


\setcounter{MaxMatrixCols}{10}

% Global Document settings
%\graphicspath{{C:/Work/Papers/Murray/Projects/AIT/ait-ahmad-murray/paper/Images}}
% \linespread{1.3} % Setting it to 1.3 = 1.5 line spacing; 1.6 = double spacing
\onehalfspacing
\setlength{\parindent}{0in}
\setlength{\parskip}{1em}

% Create arg max/min operator
\DeclareMathOperator*{\argmax}{arg\,max}
\DeclareMathOperator*{\argmin}{arg\,min}
\DeclareMathOperator{\E}{\mathbb{E}}

\newcommand{\bi}{\begin{itemize}}
\newcommand{\ei}{\end{itemize}}
\newcommand{\be}{\begin{enumerate}}
\newcommand{\ee}{\end{enumerate}}
\newcommand{\bd}{\begin{description}}
\newcommand{\ed}{\end{description}}
\newcommand{\prbf}[1]{\textbf{#1}}
\newcommand{\prit}[1]{\textit{#1}}
\newcommand{\beq}{\begin{equation}}
\newcommand{\eeq}{\end{equation}}
\newcommand{\beqa}{\begin{eqnarray}}
\newcommand{\eeqa}{\end{eqnarray}}
\newcommand{\bdm}{\begin{displaymath}}
\newcommand{\edm}{\end{displaymath}}
\newcommand{\script}[1]{\begin{cal}#1\end{cal}}
\newcommand{\citee}[1]{\citename{#1} \citeyear{#1}}
\newcommand{\h}[1]{\hat{#1}}
\newcommand{\ds}{\displaystyle}

% Remove abstract
\makeatletter
  \long\def\pprintMaketitle{\clearpage
  \iflongmktitle\if@twocolumn\let\columnwidth=\textwidth\fi\fi
  \resetTitleCounters
  \def\baselinestretch{1}%
  \printFirstPageNotes
  \begin{center}%
 \thispagestyle{pprintTitle}%
   \def\baselinestretch{1}%
    \Large\@title\par\vskip18pt
    \normalsize\elsauthors\par\vskip10pt
    \footnotesize\itshape\elsaddress\par\vskip36pt
    % \hrule\vskip12pt
    % \ifvoid\absbox\else\unvbox\absbox\par\vskip10pt\fi
    % \ifvoid\keybox\else\unvbox\keybox\par\vskip10pt\fi
    % \hrule\vskip12pt
    \end{center}%
  \gdef\thefootnote{\arabic{footnote}}%
  }
\makeatother

% Remove Elsevier preprint message
\makeatletter
\def\ps@pprintTitle{%
	\let\@oddhead\@empty
	\let\@evenhead\@empty
	\def\@oddfoot{}%
	\let\@evenfoot\@oddfoot}
\makeatother

\begin{document}
	\begin{frontmatter}
		\title{Summary of Revisions in Response to Editor and Reviewers: Implications for Determinacy with Average Inflation Targeting}
		\date{\today}
		\author[1]{Yamin Ahmad \corref{cor1}}
		\ead{ahmady@uww.edu}
		\author[2]{James Murray}
		\ead{jmurray@uwlax.edu}

		\cortext[cor1]{Corresponding author}
		\address[1]{Dept. of Economics, University of Wisconsin - Whitewater, 809 W. Starin Road, Whitewater, WI 53190, USA}
		\address[2]{Dept. of Economics, University of Wisconsin - La Crosse, 1725 State St., La Crosse, WI 54601, USA}

\end{frontmatter}

%\renewcommand{\baselinestretch}{2.0}
\renewcommand{\thefootnote}{\arabic{footnote}}%
\setcounter{page}{1}%
\setcounter{footnote}{0}%

Thank you for the opportunity to revise and resubmit this article for \textit{Economics Bulletin}. We thank the referees for all their comments, which has helped to improve the paper. Here we document a list of our comments and revisions to the paper in response to the editor and reviewers for the journal. We begin with our responses to the second referee before moving to address the questions for the first referee below.

\section*{Response to Referee 2}
We thank the Reviewer for their comments. We address their comments, point-by-point below.
\begin{enumerate}
	\item Personally, I felt that some more explanation was needed in the Results Section as to how exactly ‘determinacy’ and ‘indeterminacy’ are defined ...
	\begin{itemize}
		\item We expanded the discussion of average inflation targeting and indeterminacy in the introduction and defined them better there. We added some discussion on what we mean by determinacy versus indeterminacy. We added this discussion in the introduction, instead of the results section as the reviewer suggestion, and included an additional citation. The method we use to identify whether the model is determinant or indeterminant under various parameter calibrations is described by Sims (2002), which we cite in section 2.3. 
	\end{itemize}
	\item Should it rather be ``Panel (A) demonstrates the importance of using current or past values of inflation in the target Window”?; also: ``Shouldn’t one of those ``are necessary” and ``are also important” be omitted?
	\begin{itemize}
		\item We fixed the typos.
	\end{itemize}
\end{enumerate}

\section*{Response to Referee 1}
We thank the Reviwer for their comments. It was apparent that the reviewer was confused by some of the terminology (e.g. 'window', etc) that we used, and this may have resulted in a misunderstanding of a few points within the paper. We have attempted to address and clarify these throughout the paper where appropriate. We address their comments, point-by-point below.

\begin{enumerate}
	\item Lack of economic interpretations/intuitions. The text for Economics bulletin can reach 7 pages without tables and figures. Excluding tables and figures, your paper has approximately 5 pages. You have enough space to give more explications, interpretations and intuitions.
	\begin{itemize}
		\item In response to Reviewer \#1 (and Reviewer \#2 as described above), we expanded the discussion of average inflation targeting and indeterminacy in the introduction to take advantage of the additional space \textit{Economics Bulletin} allows.
	\end{itemize}
	\item Lack of some references: Nessén and Vestin (2005), one of the oldest references on this topis, should be cited. Other references: e.g., Piergallini (2022) also studies the stability. What is your contribution compared to this one?  
	\begin{itemize}
		\item We cite Nessen and Vestin (2005) and Piergallini (2022) in the introduction as suggested by Reviewer \#1. Nessen and Vestin (2005)  discuss average inflation targeting in the context of optimal monetary policy, with alternative viewes for the forward-looking nature of price setting (i.e. in the Phillips curve). Piergallini (2022) demonstrates how an average inflation targeting scheme with a large weight on distant observations of inflation can insure determinacy and avoid liquidity traps. Given the page limitations of the journal, we do not expand upon these contributions, but do note that these works and the others cited consider a backward-looking framework for the inflation target, and we motivate our investigation into forward-looking inflation targeting.
	\end{itemize}
	\item The paper should precise the sens of the term “window: is it a period, the value of the target or something else? term appears for the first time on page 1 in the sentence “...is the window for how the ‘average’ level of inflation is determined. It may be based purely on past values of inflation, expectations of future values for inflation, or some combination”. This explanation creates confusion et is not a very clear expression. example, ``Monetary policy targets an average value of inflation over a target window that may include backward- and forward-looking terms for inflation” on page 3. Here, what is sense of “window”? When you write “window ..include... terms”, it is not in the sense of time horizon. contrast, Mota and Fernandes (2022) also speak of “window” but it is the period over which the average is calculated. Eo and Lie (2020) have also used “target window” in the sence of time horizon. You should clearly define the term “window” and check all the text to verify if this term is used coherently.
	\begin{itemize}
		\item The term 'window' is exactly in the sense of a time horizon. We define the term `window' now in the introduction as the specific reference periods used to construct the measure of the average inflation target. Hopefully this will help to clarify any confusion caused, e.g. ``Monetary policy targets an average value of inflation over a target window that may include backward- and forward-looking terms for inflation”. 
	\end{itemize}
	\item Pages 1 and 2, in Eqs. (1)-(3), for the same variable, the notations are differents, $\pi_{t+1|t}^{e}$ in Eqs (1) and (3) but $\pi_{t+1|}^{e}$ in Eq. (2).
	\begin{itemize}
		\item Reviewer \#1 identified unnecessary use of two different notations for the same expectation, $x^e_{t+1|t}$ and $x^e_{t+1}$ for output gap expectations and $\pi^e_{t+1|t}$ and $\pi^e_{t+1}$. In general, since we allow for agents who do not have rational expectations in the model, we have to differentiate between those agents with rational expectations vs those who have na\"ive expectations given the information set at time period $t$. Thus, we changed all instances of aggregate expectations to $x^e_{t+1}$ and $\pi^e_{t+1}$, respectively, and have specifically written that $\E_t$ represents the mathematical expecations operator for those who form rational expectations after equation (2).
	\end{itemize}
	\item Page 3: ``Smaller values for $\delta_B$ can be viewed as longer backward-looking windows”. $\left(1-\delta_B\right)$ to discount the past, the effective time window is infinite although the distant past is weakly taken into account.  alternative way is to compute the average only in the limited time windows (e.g., 5 years in the past and data in more distant past is ignored). As you solve numerically the model, you can test differents time windows over which the average past inflation is computed and see if they are compatible with the dynamic stability of the economy. should explain why not use the approach of limited time window. If you continue to use the approach of “infinite time” window, then explain why the values of $\delta_B$ is a good proxy of the limited time window” as in any way, you cannot solve model analytically. Is there any facility in simulation that justify the use of $1/ \delta_B$ as the target window?
	\begin{itemize}
		\item While we have added some discussion to be more explicit on our meaning of the word "window", we have also added some motivation for our more continuous approach. Our windows are not finite windows under which an arithmetic mean is calculated, but rather infinite weighted averages, where the weight decline geometrically with time distance from the current period. The more recent past is weighted more heavily than the more distant past, and expectations of the more immediate future are weighted more heavily than expectations of the more distant future. We discuss this in more detail in the paragraph following equation (7). 
	\end{itemize}
	We address the following three points together below.
	\item Page 2, Eq. (2): Give the references where this type of rule is first used. There are multiple possible forms of the Taylor rule with backwrd looking, why choose this one? Can you briefly discuss the implications of some alternative rules?
	\item Page 3, “the nature with which the weights decline geometrically with time” lets the reader perplex about the word “nature”. The sentence may become clearer if it is rewritten as something like “that the current average inflation is a function of past inflation rates with their weights declining geometrically with time”.
	\item It seems that your paper considers for the first time “backward- and forward-looking windows”. If this is the case, you affirm this in the paper. If not, you cite the original contributions. You need to explain why the central bank uses different coefficients to compute past and expected future average inflation.
	\begin{itemize}
		\item Reviewer \#1 it their comment at the end of page 2 and page 3 discusses the forward-looking versus backward-looking weights, $\delta_F$ and $\delta_B$. In part, we felt some of this confusion arose from the earlier issue with the term ``window", which we have now clarified. As mentioned in our earlier response, and to further clear up some confusion regarding these three points above. Consequently, we added some discussion following equations (7) and (9) as to the rationale for our model structure using backwards and forwards looking windows, as well as why it is possible the Fed would use different weights for its forward looking average inflation versus backward-looking average inflation. At the top of page 3 in Reviewer \#1's comments, the reviewer suggests a ``change of notation," but this also changes the definition of the forward-looking inflation to instead be backward looking. This equation is not a correct implication of our model, nor are the mathematical results that follow.
	\end{itemize}
	\item It seems that the model cannot be solved analytically, at least for the short-run equilibrium. But it is possible to solve for the steady state and check for the consistency of different equations: is there a unique solution of equilibrium? What are the steady-state values of inflation, the output gap and the interest rate?
	\begin{itemize}
		\item Reviewer \#1 asks about the steady state of the model. The steady states are defined in the set up of the model. The steady state interest rate is $r^n = 1/\beta - 1$, as described after equation (1), steady state inflation is the Fed's target, $\pi^*$, as described after equation (3), and the steady state output gap is 0.0. This is a linearized solution to a standard New Keynesian model that is described well in Clarida, Gali, and Gertler (1999) which we cite at the beginning of Section 2.
	\end{itemize}
	\item “Figure 1 shows the regions of determinacy for different values of the forward-looking weight, $\delta_F$, depending on four other parameters in the model.” But in the figure, there are 6 parameters.
	\begin{itemize}
		\item Reviewer \#1 correctly identifies that in our line, "regions of determinacy for different values of the forward-looking weight, $\delta_F$, depending on four other parameters in the model," that we actually consider six other parameters. We fixed this, changing our description to refer to six parameters.
	\end{itemize}
	\item In Panel (C), ``When more than 40\% of agents form naïve expectations, no purely forwardlooking window for AIT leads to determinacy”. Does $\delta_F > 0.5$ signifies “purely forward looking window”?
	\begin{itemize}
		\item This comment probably stems from the earlier confusion about what constitues a window. The parameter that determines the weight put towards forward vs backwards-looking windows is $\gamma$ given by equation (5) in the paper. If $\gamma=0$, then we have purely forward-looking windows used to construct the average inflation target. The parameters $\delta_B$ and $\delta_F$ determine the appropriate duration of the equivalent fixed window for the backwards and forwards looking terms respectively. 
	\end{itemize}
	\item It is not clear why you talk about current inflation and future inflation and associate these variables with “63\% weight” and “37\% weight”.
	\begin{itemize}
		\item See comment above. This likely arose as a result of misunderstanding the role of $\gamma$. Since our backward-looking window consists of the current and past values of inflation. Since in Panel A, $\delta_B=1$, that means that the backwards-looking window only consists of the current value for inflation (see equation 6). Consequently, in that particular result, 63\% of the weight is based on the current observation of inflation and 37\% of the weight is based upon expected future inflation. We have clarified that $\delta_B=1$ in that sentence.
	\end{itemize}
	\item Page 6, “Larger response to inflation lead to more restrictive forward windows.” In Panel (D), the relation between $\delta_F$ and $\psi_\pi$ is not monotonic and at least for small values of $\psi_\pi$ there is a decreasing relation. What does it mean “restrictive”? Does it mean “smaller
	window”?
	\begin{itemize}
		\item We included a value of $\psi_\pi = 1$ within Panel (D) in the chart for greater clarification and clarified the relevant sentence in the paper. This is simply the Taylor principle. After the value of $\psi_\pi=1$, there is a monotonic relationship. It means that if the Fed puts a greater weight on inflation in the monetary policy rule, there are fewer values of $\delta_F$ that yield determinacy. In other words, the bigger the value of $\delta_F$ (as can be seen in panel D which yields determinate outcomes for the model), this implies a smaller forward window (given by $1/\delta_F$).
	\end{itemize}
	\item Page 6, “Larger responses to the output gap are necessary are also important for determinacy” There are two “are” in the sentence whose sense is not clear. Maybe you should rather said that it “increases the forward looking target window” instead of “ important for determinacy”, which is vague. The following sentence is more precise and is sufficient on its own.
	\begin{itemize}
		\item We adjusted the sentence based on the comment from Reviwer \#2. We also added greater clarity. 
	\end{itemize}
	\item Page 6, “Monetary policy persistence can play an important role”. Here, “play an important role” is an imprecise expression. More precision is necessary.
	Otherwise, this sentence can be cut.
	\begin{itemize}
		\item Following the suggestion by Reviewer \#1 to be more explicit on the role of monetary policy persistence, we changed the pair of sentences and added greater clarification. 
	\end{itemize}
	\item First, the statement is incoherent because in Figure 1, forward weight (hence forward window) depends on other parameters. In this sentence, all parameters are put on 	the same ground. Second, this affirmation is contradictory with the comment “When $\gamma > 0.63$, all possible forward-looking windows yield determine solutions” (not just greater than two years) in the graphical analysis on page 5.
	\begin{itemize}
		\item We have clarified that these are a summary of the key results based upon the benchmark values in the model. They do not all occur at the same time, but rather are trying to succintly indicate the key findings.
	\end{itemize}
	\item What is the weight? Do you talk about $\delta_B$ since in equation (6) (we observe the presence of present inflation but only average past inflations) ? The term “window” is the period over which the average is calculated, and has no direct relation with weight.
	\begin{itemize}
		\item This comment appears to be from the earlier confusion of what is a window vs weight. Please see earlier responses that address this throughout the paper.
	\end{itemize}
\end{enumerate}


\end{document}
