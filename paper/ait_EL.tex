\documentclass[english,authoryear,12pt]{elsarticle}
%%%%%%%%%%%%%%%%%%%%%%%%%%%%%%%%%%%%%%%%%%%%%%%%%%%%%%%%%%%%%%%%%%%%%%%%%%%%%%%%%%%%%%%%%%%%%%%%%%%%%%%%%%%%%%%%%%%%%%%%%%%%%%%%%%%%%%%%%%%%%%%%%%%%%%%%%%%%%%%%%%%%%%%%%%%%%%%%%%%%%%%%%%%%%%%%%%%%%%%%%%%%%%%%%%%%%%%%%%%%%%%%%%%%%%%%%%%%%%%%%%%%%%%%%%%%
\usepackage{fullpage}
\usepackage[labelfont=bf,singlelinecheck=false,aboveskip=10pt]{caption}
%\usepackage[font={scriptsize},labelfont={scriptsize},textfont={scriptsize}]{caption}
%\DeclareCaptionFormat{myformat}{#1#2\\#3}
\usepackage{amsmath,graphicx,multicol,chngpage,setspace}
\usepackage{amssymb,amsthm,amstext,amsfonts,enumerate}
\usepackage[colorlinks=true,linkcolor=black, citecolor=blue, urlcolor=blue]{hyperref}
\usepackage[top=1in, bottom=1in, left=1.0in, right=1.0in]{geometry}
\usepackage{lscape}
\usepackage{rotating}
\usepackage{color,float,multirow,array,hyphenat}
\usepackage{booktabs}
\usepackage{subcaption}
\usepackage{natbib} % already loaded by elsarticle
\usepackage{apalike}
\usepackage{babel}
\usepackage{appendix}
\usepackage{afterpage}


\setcounter{MaxMatrixCols}{10}

% Global Document settings
%\graphicspath{{C:/Work/Papers/Murray/Projects/AIT/ait-ahmad-murray/paper/Images}}
% \linespread{1.3} % Setting it to 1.3 = 1.5 line spacing; 1.6 = double spacing
\onehalfspacing
\setlength{\parindent}{0in}
\setlength{\parskip}{1em}

% Create arg max/min operator
\DeclareMathOperator*{\argmax}{arg\,max}
\DeclareMathOperator*{\argmin}{arg\,min}
\DeclareMathOperator{\E}{\mathbb{E}}

\newcommand{\bi}{\begin{itemize}}
\newcommand{\ei}{\end{itemize}}
\newcommand{\be}{\begin{enumerate}}
\newcommand{\ee}{\end{enumerate}}
\newcommand{\bd}{\begin{description}}
\newcommand{\ed}{\end{description}}
\newcommand{\prbf}[1]{\textbf{#1}}
\newcommand{\prit}[1]{\textit{#1}}
\newcommand{\beq}{\begin{equation}}
\newcommand{\eeq}{\end{equation}}
\newcommand{\beqa}{\begin{eqnarray}}
\newcommand{\eeqa}{\end{eqnarray}}
\newcommand{\bdm}{\begin{displaymath}}
\newcommand{\edm}{\end{displaymath}}
\newcommand{\script}[1]{\begin{cal}#1\end{cal}}
\newcommand{\citee}[1]{\citename{#1} \citeyear{#1}}
\newcommand{\h}[1]{\hat{#1}}
\newcommand{\ds}{\displaystyle}


% Remove Elsevier preprint message
\makeatletter
\def\ps@pprintTitle{%
	\let\@oddhead\@empty
	\let\@evenhead\@empty
	\def\@oddfoot{}%
	\let\@evenfoot\@oddfoot}
\makeatother

\begin{document}
	\begin{frontmatter}
		\title{Macroeconomic Implications of Cost Shocks With Various Average Inflation Targeting Regimes}
		\date{\today}
		\author[1]{Yamin Ahmad}
		\ead{ahmady@uww.edu}
		\author[2]{James Murray}
		\ead{jmurray@uwlax.edu}
		
		\cortext[cor1]{Corresponding author}
		\address[1]{Dept. of Economics, University of Wisconsin - Whitewater, 809 W. Starin Road, Whitewater, WI 53190, USA}
		\address[2]{Dept. of Economics, University of Wisconsin - La Crosse, 1725 State St., La Crosse, WI 53190, USA}		
	
	\date{\today}
	
	\begin{abstract}
		%\renewcommand{\baselinestretch}{1.5}
		We use a three-equation New Keynesian model with average inflation targeting (AIT) to demonstrate the impact that cost shocks have in Taylor rules with varying forward-looking and backward-looking horizons for the target average inflation.
		
		\begin{flushleft}
			{\it JEL Classification}:  \newline
			{\it Keywords}: Average Inflation Targeting, AIT
		\end{flushleft}
	\end{abstract}
	
\end{frontmatter}

%\renewcommand{\baselinestretch}{2.0}
\renewcommand{\thefootnote}{\arabic{footnote}}%
\setcounter{page}{1}%
\setcounter{footnote}{0}%


\section{\label{Intro}Introduction}
In August 2020, the Federal Reserve became the first central bank to adopt average inflation targeting (AIT) as their monetary policy framework. At the Jackson Hole symposium, Fed Chairman Jerome Powell laid out their new strategy, under which inflation could temporarily deviate in the short run from the Fed's inflation target, as long as the average level of inflation in the medium to long run remained consistent with the Fed's target. Consequently under AIT, if inflation were to consistently remain below its target for some period, it would be followed for some period where inflation would remain above its target, and vice versa. In this way, one aim of adopting an AIT framework is to help provide a nominal anchor where long-run inflation expectations are consistent with the central bank's target. 

A slew of academic research has subsequently begun to revisit the AIT framework, examining a range of issues such as the welfare implications under AIT (e.g. \citealp{budianto_average_2020}; \citealp{eo_average_2020}), to how adoption of AIT has impacted inflation expectations (e.g. \citealp{coibion_average_2020}; \citealp{hoffmann_would_2022}). A central question that pertains to the literature on the AIT framework is how the `average' level of inflation is determined. Is the average based upon past values of observed inflation? Does it depend purely on expectations of future inflation? Or is it a hybrid of the two? 

In this paper, we examine this issue within the context of a standard New Keynesian model. We evaluate how the construction of `average' inflation - from the size of the window used to determine the average value, to whether the average level of inflation is based upon backwards or forwards looking values, or even a hybrid of the two - impacts the path of inflation from an inflation shock, as well as its impact on central bank credibility.


\section{\label{Model}Model}
This paper builds upon a standard representative agent New Keynesian (RANK) model along the lines of \cite{clarida1999science} and \cite{steinsson2003optimal}. This standard model consists of infinitely lived optimizing households, monopolistically-competitive firms, and a central bank.  

\subsection{Baseline Framework}
There are three key equations of interest. The first is the dynamic IS curve that may be derived from the consumer's problem. Once the model has been log-linearized, it states that the current output gap depends on expectations of the output gap next period, and is negatively related to the difference between the ex-ante real interest rate and the natural rate of interest. This is given by the equation:
\begin{equation}\label{eq:ISe}
	x_t = x_{t+1|t}^e - \frac{1}{\sigma} \left( r_t - \pi_{t+1|t}^e -r^n_t \right) + \xi_t^{x}
\end{equation}
where in period $t$, $x_t$ denotes the output gap (given by the difference between the log of output and its natural rate), $r_t$ denotes the nominal interest rate, $\pi_t$ is the inflation rate and $r^n_t$ is the natural rate of interest. The terms $x_{t+1|t}^e$ and $\pi_{t+1|t}^e$ represent the period $t$ expectations of private sector agents on next period's output gap and inflation respectively. The preference parameter, $\sigma$ is inversely related to consumers' intertemporal elasticity of substitution and the shock term, $\xi_t^x$, represents a demand shock.

The second key equation is the New Keynesian Phillips Curve (NKPC), which states that inflation depends on the expectation of next period's inflation, and the output gap.
\begin{equation}\label{eq:PhillipsCurvee}
	\pi_t = \beta \pi_{t+1|t}^e + \kappa x_t + \xi_t^{\pi}
\end{equation}
where the preference parameter $\beta$ is the discount factor, $\kappa$ is a reduced form parameter that is inversely related to the degree of price stickiness, and the shock term, $\xi_t^\pi$, represents a possible cost shock, or stagflationary shock.

The third relationship is an interest rate rule that describes the conduct of monetary policy and how nominal interest rates are determined: 
\begin{equation}\label{eq:TaylorRule}
	r_t = \rho_r r_{t-1} + (1-\rho_r) \left( \psi_\pi \pi_t + \psi_x x_t \right) + \epsilon_t^{r}
\end{equation}
where $\rho_r$ represents the parameter that central banks use to smooth the path of interest rates. The parameters $\psi_\pi$ and $\psi_x$ represent the weight that the central bank places on current inflation relative to the output gap respectively. The shock term, $\epsilon_t^r$, denotes a monetary policy shock.

Regarding the IS and the Phillips curve equations, we will consider the possibility that a fraction $\lambda \in [0,1)$ of agents form naive expectations, so that on the aggregate, the time $t$ expectations for the future output gap and future inflation rate are given by,
\begin{equation}\label{eq:xe}
	x_{t+1|t}^e = \lambda x_t + (1-\lambda) \E_t x_{t+1},
\end{equation}
\begin{equation}\label{eq:pie}
	\pi_{t+1|t}^e = \lambda \pi_t + (1-\lambda) \E_t \pi_{t+1},
\end{equation}
where $x_{t+1|t}^e$ denotes expectation made at time $t$ regarding the value $x$ in time $t+1$. When $\lambda=0$, both $x_{t+1|t}^e = \E_t x_{t+1}$ and $\pi_{t+1|t}^e=\E_t \pi_{t+1}$. In this case, expectations are fully rational. When $\lambda$ is greater than zero, this implies that a cost shock can have a larger impact on inflation. Only a fraction of agents in the economy use the information that inflation will come back down once the central bank responds with an increase in the interest rate. This effect can be even larger when the central bank employs average inflation targeting, allowing inflation to remain above its target from some length of time.

Finally, in the baseline model, we assume that demand shocks and cost shocks have persistence. The innovations $\xi_t^x$ and $\xi_t^\pi$ evolve according to:
\begin{equation}\label{eq:demandshock}
	\xi_t^x = \rho_x \xi_{t-1}^x + \epsilon_t^x,
\end{equation}
\begin{equation}\label{eq:costshock}
	\xi_t^\pi = \rho_\pi \xi_{t-1}^\pi + \epsilon_t^\pi,
\end{equation}
where $\rho_x\in [0,1)$ and $\rho_\pi \in [0,1)$ capture the persistence of the shocks and $\epsilon_t^x$ and $\epsilon_t^\pi$ are independently and identically distributed with zero mean and variances, $\sigma_x^2$ and $\sigma_\pi^2$, respectively.

\subsection{Average Inflation Targeting}

Under average inflation targeting (AIT), than central bank targets an (potentially weighted) average value of inflation instead of the current inflation rate. The monetary policy rule under AIT may be described as: 
\begin{equation}\label{eq:TaylorRuleAIT}
	r_t = \rho_r r_{t-1} + (1-\rho_r) \left( \psi_\pi \pi_t^A + \psi_x x_t \right) + \xi_t^{r},
\end{equation}
which is simply equation (\ref{eq:TaylorRule}) with $\pi_t^A$ in substitution for $\pi_t$, and $\pi_t^A$ is equal to some average inflation rate over some period of time. The U.S. Federal Reserve has not been specific on how long such a window may be, and whether that window is forward-looking, backward-looking, or both. For example, the average inflation target could possibly look like:
\begin{equation}\label{eq:ait}
	\pi_t^A = \frac{m+1}{m+n+1}\sum_{j=0}^{m} \pi_{t-j} + \frac{n}{m+n+1}\sum_{j=1}^{n} \E_t \pi_{t+j},
\end{equation}
where the window looks backward $m$ periods, includes the current period (hence the $m+1$), and looks forward $n$ periods.  This arithmetic mean gives an equal weight to all observations within the window. We assume the central bank has rational expectations and that forward-looking expectation uses the rational expectations operator. It is both mathematically convenient and plausible to consider that the target be a weighted average, where observations closer to the current period are weighted more heavily than observations in the farther future or farther past. 

We consider that the target average inflation rate, $\pi_t^A$, which appears in the monetary policy rule in equation (\ref{eq:TaylorRuleAIT}), is the following weighted average of the backward-looking and the forward-looking average,
\begin{equation}
	\pi_t^A = \gamma \pi_t^B + (1-\gamma) \pi_t^F,
\end{equation}
where $\gamma \in [0,1]$ is the relative weight given to past inflation versus expected future inflation.

Here $\pi_t^B$ is a measure of the backward-looking weighted average inflation described by:
\begin{equation}\label{eq:backward}
	\pi_t^B = \delta_B \pi_t + (1-\delta_B) \pi_{t-1}^B,
\end{equation}
where the weight $\delta_B \in (0,1)$ given to the most recent observation. Note that the current value for inflation, $\pi_t$, is included in this ``backward-looking" window. The backward-looking average can be thought of everything in the time $t$ information set for the central bank. Repeated substitution reveals $\pi_t^B$ is a weighted average with weights declining geometrically with time,
\begin{equation}\label{eq:backward_all}
	\pi_t^B = \delta_B \sum_{j=0}^{\infty} (1-\delta_B)^j \pi_{t-j},
\end{equation}
where the weight on an observation $j$ periods in the past is given by, $\delta_B (1-\delta_B)^j$. The weights sum to one and converge to zero as observations are more distant. Smaller values for $\delta_B$ can be viewed as longer backward-looking windows for average inflation targeting. With an equally-weighted mean over a finite window of size $(m+1)$, the weight on every observation is $1/(m+1)$. The weight of $\delta_B$ could then be though of as approximating the monetary policy behavior of using a finite window of length $1 / \delta_B$ periods.

Similarly, $\pi_t^F$ is a forward-looking expected average of future inflation given by:
\begin{equation}\label{eq:forward}
	\pi_t^F = \delta_F E_t \pi_{t+1} + (1-\delta_F) E_t \pi_{t+1}^F,
\end{equation}
where $\delta_F \in (0,1)$ is the weight on $E_t \pi_{t+1}$, or the weight given to the soonest expected future inflation. Note that the forward-looking expectation does not include the current value for inflation. Note that the expression does not include the current inflation rate (which was included in the backward-looking average). The forward-looking average is a sum of only expected future outcomes.

Repeated substitution of equation (\ref{eq:forward}) leads to the following expression,
\begin{equation}\label{eq:forward_all}
	\pi_t^F = \delta_F \sum_{j=0}^{\infty} (1-\delta_F)^j E_t \pi_{t+1+j},
\end{equation}
which reveals the weight given to an expected inflation rate $j$ periods in the future is equal to $\delta_F (1-\delta_F)^{j-1}$. The weights decline geometrically with the distance into the future, converging to zero and all summing to one.



The continuous nature of the forward-looking and backward-looking weighted averages are both mathematically convenient and arguably more realistic than rolling-window arithmetic mean. Equations (\ref{eq:backward}) and (\ref{eq:forward}) are difference equations over a single period. The parameters, $\delta_B$ and $\delta_F$ can be varies on a continuous scale, instead of making discrete choices for rolling window lengths. A standard Taylor-type rule, like equation (\ref{eq:TaylorRule}), is also a special case of the larger model, with $\gamma=1.0$ and $\delta_B=1.0$. By decreasing these parameters below 1.0, we can think of larger deviations from the standard monetary policy rule.

It may be more realistic to suppose that the central bank puts more weight on more recent past observations than on farther observations. It may be more realistic to suppose too that the central bank puts more weight on future observations in the near future, where inflation forecasts may be more reliable, than on expectations for the farther future.

\subsection{Full Model}

The full model is given by the following ten equations: IS curve, equation (\ref{eq:ISe}); the Phillips curve, equation (\ref{eq:PhillipsCurvee}); the definition for the possibly non-rational expected future output gap and future inflation, equations (\ref{eq:xe}) and (\ref{eq:pie}), respectively; the evolution for the IS and Phillips curve shocks, equations (\ref{eq:demandshock}) and (\ref{eq:costshock}), respectively; the Taylor rule with average inflation targeting, equation (\ref{eq:TaylorRuleAIT}); the average inflation target, equation (\ref{eq:ait}); the backward-looking average, equation (\ref{eq:backward}); and the forward-looking average, equation (\ref{eq:forward}). There are ten endogenous variables that include the output gap, $x_t$; the inflation rate, $\pi_t$; the interest rate, $r_t$; expected future output gap, $x_{t+1|t}^e$; expected future inflation, $\pi_{t+1|t}^e$; the average inflation target, $\pi_t^A$; the backward-looking average inflation rate, $\pi_t^B$; the forward-looking average inflation rate, $\pi_t^F$; and the IS and Phillips curve shocks, $\xi_t^x$ and $\xi_t^\pi$, respectively.

We calibrate the model using the parameters in Table \ref{tb:parms}. In the next section we consider alternative calibrations for the average inflation targeting parameters and the degree of rational versus naive expectations. Table \ref{tb:parmsvary} shows variations on those parameters that are considered. For a given parameterization, we solve the linear model using the method in \cite{sims_solving_2002}.

\begin{table}[htp]
	\caption{Parameter Calibrations}\label{tb:parms}
	\begin{center}
		\begin{tabular}{lcr} \hline
			Description & Parameter & Value \\ \hline
			Discount rate (Quarterly) & $\beta$ & 0.99 \\
			Inverse intertemporal elasticity & $\sigma$ & 2.0 \\
			Natural rate of interest & $r^n$ & 0.02 \\
			Phillips curve coefficient & $\kappa$ & 0.1 \\
			Monetary policy persistence & $\rho_r$ & 0.7 \\
			Monetary policy response to average inflation & $\psi_\pi$ & 1.5 \\
			Monetary policy response to output gap & $\psi_x$ & 0.5 \\
			Persistence of demand shock & $\rho_x$ & 0.7 \\
			Persistence of inflation shock & $\rho_\pi$ & 0.7 \\
			Standard deviation demand shock & $\sigma_x$ & 0.25 \\
			Standard deviation inflation shock & $\sigma_\pi$ & 0.25 \\
			\hline
		\end{tabular}
	\end{center}
\end{table}

\begin{table}\caption{Parameter Variations Considered}\label{tb:parmsvary}
	\begin{center}
		\begin{tabular}{lccc}
			Description & Parameter & Baseline & Values Considered \\ \hline
			Backward-looking weight for AIT & $\delta_B$ & 0.25 & $\left\{0.125, 0.25, 1.0 \right\}$ \\  [0.25pc]
			Forward-looking weight for AIT & $\delta_F$ & 0.25 & $\left\{0.125, 0.25, 1.0 \right\}$ \\  [0.25pc]
			AIT weight on past inflation & $\gamma$ & $\left\{ 0.0, 1.0 \right\}$ & $\left\{ 0.0, 0.25, 0.75, 1.0\right\}$ \\  [0.25pc]
			Weight on naive expectations & $\lambda$ & 0.0 & $\left\{ 0.0, 0.25, 0.75 \right\}$ \\ [0.25pc]
			\hline
		\end{tabular}
	\end{center}
\end{table}

\section{Results}



\bibliographystyle{apalike}
\bibliography{ait}

\end{document}
